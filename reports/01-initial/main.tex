\documentclass[a4paper,english,10pt,NET]{tumarticle}

% TUM packages
\usepackage{tumfonts}
\usepackage{tumarticle}
\usepackage{tumlocale}
\usepackage{tumcmd}

\usepackage{graphicx} % Required for inserting images
\usepackage{csquotes}
\usepackage{hyperref}
\usepackage{xspace}
\usepackage{booktabs}
\usepackage{siunitx}
\usepackage{todo}

\renewcommand{\eg}{e.g.\xspace} % already defined in tumcmd but without xspace
\renewcommand{\ie}{i.e.\xspace} % already defined in tumcmd but without xspace
\newcommand{\cf}{c.f.\xspace}
\newcommand{\tos}{$\to$\xspace}

% Set document title. If no language is supplied, the document language is
% assumed.
\title{P2Psec Project -- Initial Report}
\author{Fabio Gaiba, Lukas Heindl}
\date{May 21st, 2024}

\begin{document}

\maketitle
\thispagestyle{tumarticle}

\section{Introduction}

We decided to name the team \textbf{Peer2Whisper}, like the peers are whispering to each other to spread information in the network. The team members are Fabio Gaiba and Lukas Heindl.

\Todo{Does the team name sound cool? -- maybe Whisper2Peer? but both sound good for me}

\Todo{They mention "which implementation" we chose, what's that? Gossip or somehting else? -- yes I think they mean the module we are working on, "the module which we want to implement"}


\section{First Decisions}
\subsection{Ecosystem}
\subsubsection{Language}

We opted to use Golang to develop the module for multiple reasons:

\begin{enumerate}
	\item The language has good native support for asynchronous programming 
	\item It comes with a good set of developing tools (packet manager, code formatter/linter)
	\item It is statically typed  
	\item Lukas has good experience with it and Fabio is interested in learning it.
	\item It's what the \enquote{cool kids (xD)} use
\end{enumerate}

\Todo{Check if correct (I think I can use bullet points this way) -- there is no built-in linter, not sure if we should use an enumeration if they say no bullet points. Might look like we're stalling for space}

\subsubsection{OS}
We use Linux as it is the easier choice for developing (almost) any software. Moreover we find the UNIX philosophy very intuitive and we are fond of the free software principles in general. We believe that the open source nature of the Linux kernel makes it a more secure, reliable and efficient alternative than any proprietary OS.

Fabio's Distro:
He wanted to experiment with the declarative definition of the system, particularly in setting up a customized desktop environment in an easy-to-reproduce manner, and with the separation of developing packages from system packages. He is still very new to Nix's principles.

Lukas' Distro:
Lukas will be using Arch Linux as distribution, as this is what he is using as day to day production system.

\Todo{Check (and adjust) the common statement about Linux}

\subsubsection{Build-system}
With Golang the most natural decision for the build system is to simply use the utility that comes with Golang and is included in the Golang ecosystem.
The tooling of Golang also already includes dependency management.
In case we need a more elaborate build-chain for \eg generating protobuf/capnproto code before building the binary, we may use a simple \texttt{Makefile}.

\subsubsection{Quality of software}
% TODO Fabio

Golang offers a built-in support for unit testing. To test \verb|example.go| it will be sufficient to write the unit test in \verb|example_test.go| and then run \verb|go test| to run all unit tests. As far as how will those unit test be written, we want to take a simple approach with \textit{Black Box Testing} (so just looking at the input/output pairs) and use the \textit{AAA} pattern (Arrange, Act and Assert).

To ensure Code Quality, we will use built-in go tools as \verb|go fmt| for formatting and \verb|go vet| as a static code checker. We will also take advantage of slightly more powerful static checker as \href{https://staticcheck.dev/}{staticcheck} and \href{https://github.com/golangci/golangci-lint}{golangci-lint}.

We also think it would be interesting to look into \href{https://git-scm.com/book/en/v2/Customizing-Git-Git-Hooks}{git} hooks to perform automatic local testing before pushing (or merging to main). We also looked into Gitlab CI to run automatic testing after pushing the repo but it does not seem to be enabled.  

\Todo{check git hooks part -- not sure if checking before merging is possible, need to revisit the git docs}
\Todo{is golangci-lint a static checker?}

% https://go.dev/doc/tutorial/add-a-test

\subsubsection{Libraries}
The most important libraries, namely
\href{https://pkg.go.dev/log}{log}
/
\href{https://pkg.go.dev/log/slog@go1.22.2}{slog},
\href{https://pkg.go.dev/net}{networking} (tcp+udp)
and
\href{https://pkg.go.dev/crypto/rand}{cryptographic randomness},
are already included in the stdlib of Golang.
To get slightly nicer logging (mainly for the coloring) we plan to use \href{github.com/lmittmann/tint}{tint}.
We also need to be able to parse a configuration file in the ini-format (like we already did in our client for the registration).
For that we plan on using the library \href{gopkg.in/ini.v1}{ini} once again.
The path to this configuration file is passed via a commandline argument.
Using a library (\href{https://github.com/alexflint/go-arg}{go-arg}) for parsing the commandline arguments provides us with a very flexible way to parse commandline arguments.
This way if we decide later on to add other commandline arguments, this will be an easy addition.

Coding data (de)serialization by hand is cumbersome and error prone.
For this reason there is a variety of (mostly language agnostic) formats/libraries like \href{https://protobuf.dev/}{protobuf}/\href{https://flatbuffers.dev/}{FlatBuffers} and \href{https://capnproto.org/}{capnproto} out there.
Based on our current research, it looks like capnproto looks nice and also comes with with a better performance compared to the classical protobuf.
Also the developers of capnproto state it is even safe to use with a malicious peer which makes it a perfect fit for our project.
Nonetheless the developers of the Golang capnproto library note the capnproto specification is not fully finalized and the library should be considered beta software until then.
Thus, we may switch to protobuf if capnproto does not work as expected.
\Todo{think about stuff like protobuf}

\section{Legal matters}
% TODO Fabio

\Todo{I don't know if we should cite. If yes, do you know if can use BibiTex?}

We chose to use a GPLv3 license.
We use quite a lot of free software and appreciate a lot the open source community. We believe that software should be distributed with the right of using it as you want and modify it as you want. Moreover, we don't want free software to be used to produce proprietary software. A copyleft license is the best fit for our project, not necessarily because we anticipate real-world usage of it, but as a matter of principle.

%https://www.gnu.org/licenses/quick-guide-gplv3.html

\section{Previous experience}
% TODO Fabio/Lukas

Fabio has participated in the QUIC project for the Advanced Computer Network course, using Golang as the primary language. The project, while not extensive, involved implementing both a client and a server using the QUIC library developed in Go, as well as utilizing examples from the library. Additionally, he made several contributions to a \href{https://github.com/csunibo/polleg}{Go web server for a student project}.

Lukas has been participated in the same QUIC project using Golang as language of choice.
In addition he has been working on multiple side-projects like a chat-bot for the Signal Messenger or gathering data from various smart-home-like devices using Golang in this projects.

\Todo{We both did QUIC project, we could merge that}

\section{Workload distribution}
% TODO Fabio

The goal is to achieve an equal distribution of the workload between the two students. We take into account that Lukas is far more proficient in Golang, especially when it comes to using and interacting with libraries. Hence, we expect Fabio to write fewer lines of code within the same amount of time. Moreover, we anticipate Lukas being involved in peer programming sessions where he will assist Fabio, so his workload might be larger than what is visible. To balance this, Fabio will take on a larger role in report writing and in studying and designing the protocol. It's important to emphasize that this is just a forecast, and the actual workload distribution may be quite different in the end.

\Todo{Check if this makes sense and if we could/should go more into technical details}

\todos

\end{document}
